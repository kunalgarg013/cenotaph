\chapter{Measurement of $K^{*\pm}$ production in pp collisions}
\label{cap:5}


\vspace*{2cm}
$K^{*\pm}$ is a resonance particle with a small lifetime ($\sim$ 4~\fmc), comparable to that of the fireball which is produced during the heavy ion collision. Due to its short lifetime, it can be used to study the re-scattering and regeneration effects. $K^{*\pm}$ can provide the information regarding strangeness enhancement as it contains a strange quark. Measurements of $K^{*\pm}$  in pp collisions can be used as a baseline to study the PbPb collisions at the LHC energy and to provide a reference for tuning event generators.  


\section{$K^{*\pm}$  reconstruction in pp collisions}
\label{par:5.1}

General idea behind the reconstruction procedure

\subsection{Data sample and event selection}
\label{par:5.1a}
Details of the data used and event selection criteria applied

\subsection{Primary pion selection}
\label{par:5.1b} 
Selection cuts applied to the primary pion daughters of $K^{*\pm}$ 

 
\subsection{\VZERO~selection}
\label{par:5.1c} 
Selection criteria for $K^{0}_{S}$ 

\subsection{Signal Extraction}
\label{par:5.1d}
Signal extraction procedure including estimation of background and normalisation

\subsection{Raw Yield Estimation}
\label{par:5.1e} 

Fitting procedure and fit function details

\section{Efficiency Correction}
\label{par:5.2}
We use Monte Carlo to estimate the detector acceptance x efficiency


\section{Systematic Uncertainties estimation}
\label{par:5.3}
Estimation of systematic uncertainties

\section{\simplekstarch~ transverse momentum spectrum}
\label{par:5.4} 

Final \pT spectra obtained















 
