\clearpage
\markboth{\spacedlowsmallcaps{Conclusions}}{\spacedlowsmallcaps{Conclusions}}
\chapter*{Conclusions}
\addcontentsline{toc}{chapter}{Conclusions}

In this thesis the measurement of the multi-strange baryon production in \PbPb\ and \pp\ collisions at the centre-of-mass energy of \mbox{$2.76$ TeV} using the \mbox{ALICE} apparatus have been presented. The cascade identification technique, based on the topological reconstruction of weak decays into charged particles, has been described. Such a technique is very effective thanks to the excellent particle identification and tracking capability of the ALICE central barrel detectors.

The measurements of the strangeness enhancement for the \csi\ and \om\ at the LHC energy have been presented and compared with the lower energies measurements obtained by the NA57 and STAR Collaborations. The enhancements are larger than unity for all the particles. They increase with the strangeness content of the particle, showing the hierarchy already observed at lower energies and is also consistent with the picture of enhanced \ssbar\ pair production in a hot and dense partonic (deconfined) medium. The centrality dependence shows that the multi-strange particle yields grow faster than linearly with \avNpart, at least up to the three most central classes (\mbox{\Npart\ $>$ $100-150$}), where there are indications of a possible saturation of the enhancements. Compared to measurements at lower energies, the enhancements are found to decrease as the centre-of-mass energy of the collision increases, continuing the trend established at SPS and between SPS and RHIC. The historical role of the strangeness enhancement measurement within the context of the QGP study has been emphasized, focusing on a possible explanation of the energy dependence of this observable as being due to the gradual reduction of the canonical suppression mechanism in the proton-proton system. 

The hyperon-to-pion ratios as a function of  \avNpart , both in \NN\ and \pp\ collisions, from the ALICE and STAR measurements, have been proposed as an alternative way to look at the strangeness enhancement. They indicate that different factors contribute to the evolution of the enhancements with centrality. Indeed the relative production of strangeness in \pp\ collisions is larger than at lower energies. In addition, the enhancements are seen to be in part the result of a general relative increase of multiplicity at mid-rapidity, not entirely related to strangeness. The increase in the hyperon-to-pion ratios in \NN\ relative to \pp\ is indeed about half that of the usual enhancement ratio as defined by the participant-scaled yields.

The transverse momentum spectra for the \csim , \csip , \omm\ and \omp\ in \PbPb\ collisions have been compared with hydrodynamic model predictions finding that the best agreement is obtained with the Krak�w and EPOS models, with the latter covering a wider \pt\ range. In addition, the \pt\ spectra in \pp\ collisions for \csim , \csip\ and \omm +\omp\ have been compared with predictions by recent PYTHIA tunes; both the \mbox{Perugia $2011$} and \mbox{Perugia $2012$} tunes underestimate the yields for the cascades.

The nuclear modification factors for the \csi\ and \om\  have been calculated in five centrality classes and compared with the corresponding factors for $\pi^{\pm}$, $K^{\pm}$, $p$ and $\phi$ also measured by the ALICE Collaboration. The \csi\ seems to follow the same behaviour of the protons at high \pt\ to indicate that progenitor partons of the two baryons show similar energy loss, while the \om\ seems to be strongly affected by the strangeness enhancement, showing an \raa\ larger than unity. At mid-\pt , indications of mass-ordering between the different baryons (and mesons) are present. The higher \pt\ reach for all the particles compared to the spectra at RHIC and the nuclear modification factor for the \om , not even measured at lower energies, make these preliminary results uniquely interesting: the observed features provide crucial constraints for the various energy loss models. The analysis is expected to be finalized in the next months and will bring the corresponding results to publication.

\newpage
\clearpage



