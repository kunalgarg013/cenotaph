\clearpage
\markboth{\spacedlowsmallcaps{Introduction}}{\spacedlowsmallcaps{Introduction}}
\chapter*{Introduction}
\addcontentsline{toc}{chapter}{Introduction}
The present thesis is the result of my Ph.D. research programme as member of the ALICE Collaboration at the CERN (European Organization for Nuclear Research) laboratories. The main interest of this high energy physics experiment is connected to the study \textit{strong nuclear} force which is one of the four fundamental interaction forces. On the nuclear dimension scale (\mbox{$1$-$3$ fm}) this is the force that binds protons and neutrons (nucleons) together to form the nuclei of an atom; on a even smaller scale (less than about \mbox{$0.8$ fm}, the radius of a nucleon), it is the force that holds quarks together to form protons, neutrons and other particles, all called hadrons. 


ALICE experiment is designed to probe the characteristics of the nuclear matter phase diagram. \textbf{Quantum Chromodynamics} (QCD) predicts the existence of a state of matter called \emph{Quark-Gluon Plasma} (QGP),  which consists of asymptotically free strong-interacting quarks and gluons, which are ordinarily confined by colour confinement inside atomic nuclei or other hadrons at extremely high temperatures and densities. This is the state of matter that is believed to have existed in the early stages of the evolution of our universe. We can recreate QGP in high energy collisions involving ions and protons. \textit{maybe an image of phase diagram}


In the analysis reported in this document the production rates of charged multi-strange baryons, \csi\ and \om\ and their antiparticles have been measured in \PbPb\ and \pp\ collisions at the same energy in order to study the behaviour of one of the first proposed signatures for the QGP formation, the \emph{strangeness enhancement}. The original prediction that the strange quark would be produced with a higher probability in a QGP scenario with respect to that expected in a pure hadron gas scenario (as the one thought to be created in \pp\ collisions) was confirmed in measurements at lower energies. These studies can now be revisited at the much higher LHC energy, where results on strangeness enhancement and their comparisons with lower energy measurements can help to clarify the full picture. 

In addition, the study of the so called nuclear modification factor as a function of transverse momentum (\pt) and the study of multi-strange baryon production in \pp\ collisions can give insight into their production mechanism. In particular, the large transverse momentum range covered by the identification techniques adopted in the ALICE experiment provides the possibility of studying the competition between the hard mechanism (fragmentation) and soft mechanism (coalescence) in the different \pt\ regions.

This thesis is divided in seven chapters. 

\begin{description}
\item[Chapter \ref{cap:1}] The general physics context,  with an introduction to QCD and the connection with the idea of the QGP, obtained at high density and temperature, is given in this first chapter. In addition a general description of the fundamental characteristics of heavy-ion collision and the time evolution of the created system are presented. 
\item[Chapter \ref{cap:2}] This chapter is focused on the theoretical description of the strangeness production mechanisms (within the thermodynamical description) and the original idea of strangeness enhancement. Furthermore, results on the nuclear modification factor and the strangeness enhancement obtained in lower energy experiments are shown.
\item[Chapter \ref{cap:3}] In this chapter the detection capabilities of the ALICE apparatus are given. Moreover, the details on the different steps necessary to convert the electronic signals from the detectors into data suitable for analysis are presented.
\item[Chapter \ref{cap:4}] Here, the identification technique, based on the topological reconstruction of the weak decay of the multi-strange baryons, is described. Details on the difference between the two systems produced during the collision in \PbPb\ and \pp\ are discussed.
\item[Chapters \ref{cap:5} and \ref{cap:6}] In these two chapters all the needed steps to measure the production rates of multi-strange baryons, in \PbPb\ and \pp\ collisions respectively, are detailed.
\item[Chapter \ref{cap:7}] In this last chapter the physical results are presented. Transverse momentum spectra are first compared to model predictions. Then, results on the strangeness enhancement and the nuclear modification factors for the multi-strange baryons at the LHC energy are presented and discussed.
\end{description}



        