\clearpage
\markboth{\spacedlowsmallcaps{Introduction}}{\spacedlowsmallcaps{Introduction}}
\chapter*{Introduction}
\addcontentsline{toc}{chapter}{Introduction}
The present thesis is the result of my Ph.D. research programme as member of the ALICE Collaboration at the CERN (European Organization for Nuclear Research) laboratories. The main interest of this high energy physics experiment is connected to the study \textit{strong nuclear} force which is one of the four fundamental interaction forces. On the nuclear dimension scale (\mbox{$1$-$3$ fm}) this is the force that binds protons and neutrons (nucleons) together to form the nuclei of an atom; on a even smaller scale (less than about \mbox{$0.8$ fm}, the radius of a nucleon), it is the force that holds quarks together to form protons, neutrons and other particles, all called hadrons. 


This thesis is divided in seven chapters. 

\begin{description}
\item[Chapter \ref{cap:1}] The general physics context,  with an introduction to QCD and the connection with the idea of the QGP, obtained at high density and temperature, is given in this first chapter. In addition a general description of the fundamental characteristics of heavy-ion collision and the time evolution of the created system are presented. 
\item[Chapter \ref{cap:2}] This chapter is focused on the theoretical description of the strangeness production mechanisms (within the thermodynamical description) and the original idea of strangeness enhancement. Furthermore, results on the nuclear modification factor and the strangeness enhancement obtained in lower energy experiments are shown.
\item[Chapter \ref{cap:3}] In this chapter the detection capabilities of the ALICE apparatus are given. Moreover, the details on the different steps necessary to convert the electronic signals from the detectors into data suitable for analysis are presented.
\item[Chapter \ref{cap:4}]
\item[Chapters \ref{cap:5} 
\item[Chapters \ref{cap:6} 
\item[Chapter \ref{cap:7}]
\item[Chapter \ref{cap:8}]


\end{description}



        